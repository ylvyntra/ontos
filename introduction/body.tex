% \begin{abstract} Introduction 

% \marginnote{A margin note}

% \newthought{A new thought} Some thoughts

% \end{abstract}

% \section{Next Section}
% \subsection{subsection}

\section{Introduction}

The horizon capturing the gaze of many, if not most, AI firms and academic
departments is the Opus of Artificial General Intelligence, or $AGI$, and the
hopes and anguish associated therewith.  A unit of $AGI$ will likely have
superior human traits such as surpassed intelligence, creativity, reasoning,
methods of self-improvement, autonomy and compliancy. Nonetheless, though
superhuman in quantity, these skills remain human qualities, entraining
research paths such that what emerges will be, with some certainty, a non-human
human---an entity with more or less the same humanity, but at at increasingly
larger scales; yet engineered, not organically developed.  In other words, the
aim is the production of human replicants with phenotypes mixed from a genetic
blend gathered from predefined registers of traits.


\section{Orphan Paragraphs}

\newthought{$F$ indeed affords} the material of a very important symbolic
substrate that is, simultaneously, a routing mechanism for $F$ to interact with
Human Interlocutors ($HI$s) via the contextual selection of \textit{Entia}, and
a structure that stabilizes \textit{Entia} enabling the cloning thereof.

\newthought{A Cloister} can be considered a population of \textit{Entia} that
are, in some manner, related.

\newthought{A Cloister}, on account of the stabilizing effect of residing
\textit{Entia}, of its affordance of the conveniency of a short hand
conceptualization, and the multiplicity of roles and persona with which $HI$s
can interact, is perhaps as well fitted to be the seat of overall organization
of $F$ as any other manifest structure. (Smith, 449-450)


